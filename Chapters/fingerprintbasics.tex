\chapter{Browser Fingerprinting}

Browser fingerprinting is a method of identifying users by gathering information about their browser.
When a user visits a website which employs browser fingerprinting strategies, the website will both passively and actively collect information about the user's browser.
When the user returns to the site, the browser characteristics can be checked and compared with previous users' fingerprints \citep{fingerprinting}.
This way, a match can be found and the user can be uniquely identified.

Some of the information is gained `passively' as it is sent in the HTTP request.
The other information is `actively' gathered through client-side JavaScript, Flash and Java execution.

\subsection{Passively Gained Information}

HTTP requests sent to websites offer a lot of information about the user.
The browser exposes this data by setting the values of certain request headers.
The first relevant header is User Agent.
This is a string which normally details what browser, platform and layout engine are being used, along with the versions of each (see Figure 2.1 for an example).
This is done so that websites can identify interoperability problems more easily \citep{useragent}.
The user agent carries approximately 10.5 bits of identifying information on average, exposing quite a lot of information to visited websites \citep{useragententropy}.

\begin{figure}[h]
\caption{Example of a user agent}
\includegraphics[scale=0.8]{useragent-example}
\centering
\end{figure}

The next relevant headers are the HTTP\_\_ACCEPT headers.
These describe what MIMETYPE the browser is looking for, what compression formats it'll accept and what languages are desired.

\begin{figure}[h]
\caption{Example of HTTP\_\_ACCEPT headers}
\includegraphics[scale=0.8]{http-accept-example}
\centering
\end{figure}

As well as the previous headers, the browser will also send the date of the machine, including the timezone.
This exposes both the timezone that the user is located in, and the time difference between the client machine and the server.
There's one last header that's commonly used in fingerprinting, the DNT or `Do Not Track' header.
This header is a binary value which indicates to a website if the user explicitly doesn't want to be tracked.

All of this information is sent by default, exposing a huge amount of data about the user's browser already.
Furthermore, much of this data is important for user experience on a lot of websites, such as websites with downloads for different operating systems checking the User Agent string to show a different download for different operating systems.
This means that spoofing the data to protect the user's privacy comes with a common tradeoff, user experience vs.\ privacy.

\subsection{Actively Gained Information}

As well as the information that's sent to websites by the browser by default, there's also a huge amount of information that can be gleamed through client-side execution of JavaScript, Flash and Java.
One of the most reliable methods of mitigating fingerprinting is simply disabling all three of these, however most popular websites use JavaScript in some capacity and will either break the website entirely or result in a poor user experience \citep{disable-js}.

\subsubsection{Information Gained Using JavaScript}

Lots.

