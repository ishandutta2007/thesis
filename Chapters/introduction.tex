\chapter{Introduction}

As both users and browsers become more aware and wary of cookies -- the web technology which allows users to be identified for advertising purposes, it is becoming more common to avoid cookies, either by disabling them or clearing them more often than they used to.
Companies in turn have started experimenting with new tracking methods that don't require cookies to identify users.

One of these methods is called `Machine Fingerprinting', a technique that observes as many characteristics of the user's web browser and machine as possible such as what plugins are installed, screen size, time zone and other such features.
Most of this information is given freely by the browser itself, while some is actively retrieved by the websites using these techniques.
Once enough information is known about a browser, the user of that browser can be uniquely identified.

This project explores the techniques that are used for fingerprinting as well as what user information is available to these fingerprinters. Using this knowledge, a web extension is created that aims to obfuscate and to prevent the transmission of important characteristics.

This report outlines the background information required to understand why fingerprinting is being used as a method of identification and the fundamentals of how browser fingerprinting works.
It then explores the possible strategies that can be used to mitigate fingerprinting, and the design and implementation of a browser extension that attempts to do just this, along with the problems and difficulties encountered during the course of the project.

