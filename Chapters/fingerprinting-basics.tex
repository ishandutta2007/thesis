\chapter{Background: Browser Fingerprinting}

Browser fingerprinting is a method of identifying users by gathering information about their browser.
When a user visits a website which employs browser fingerprinting strategies, the website will both passively and actively collect information about the user's browser.
When the user returns to the site, the browser characteristics can be checked and compared with previous users' fingerprints \citep{fingerprinting}.
This way, a match can be found and the user can be uniquely identified.

Some of the information is gained `passively' as it is sent in the HTTP request.
The other information is `actively' gathered through client-side JavaScript, Flash and Java execution.

\subsection{Passively Gained Information}

HTTP requests sent to websites offer a lot of information about the user.
The browser exposes this data by setting the values of certain request headers.
The first relevant header is User Agent.
This is a string which normally details what browser, platform and layout engine are being used, along with the versions of each (see Figure~\ref{fig:useragent-example} for an example).
This is done so that websites can identify interoperability problems more easily \citep{useragent}.
The user agent carries approximately 10.5 bits of identifying information on average, exposing quite a lot of information to visited websites \citep{useragententropy}.

\begin{figure}[h]
\caption{Example of a user agent}
\includegraphics[scale=0.8]{useragent-example}
\centering
\label{fig:useragent-example}
\end{figure}

The next relevant headers are the HTTP\_\_ACCEPT headers.
These describe what MIMETYPE the browser is looking for, what compression formats it'll accept and what languages are desired.

\begin{figure}[h]
\caption{Example of HTTP\_\_ACCEPT headers}
\includegraphics[scale=0.8]{http-accept-example}
\centering
\end{figure}

As well as the previous headers, the browser will also send the date of the machine, including the timezone.
This exposes both the timezone that the user is located in, and the time difference between the client machine and the server.
There's one last header that's commonly used in fingerprinting, the DNT or `Do Not Track' header.
This header is a binary value which indicates to a website if the user explicitly doesn't want to be tracked.

All of this information is sent by default, exposing a huge amount of data about the user's browser already.
Furthermore, much of this data is important for user experience on a lot of websites, such as websites with downloads for different operating systems checking the User Agent string to show a different download for different operating systems.
This means that spoofing the data to protect the user's privacy comes with a common tradeoff, user experience vs.\ privacy.

\subsection{Actively Gained Information}

As well as the information that's sent to websites by the browser by default, there's also a huge amount of information that can be gleamed through client-side execution of JavaScript, Flash and Java.
One of the most reliable methods of mitigating fingerprinting is simply disabling all three of these, however most popular websites use JavaScript in some capacity and will either break the website entirely or result in a poor user experience \citep{disable-js}.

\subsubsection{Information Gained Using JavaScript}

JavaScript exposes a huge amount of information to fingerprinters.
A lot of the information is stored in the \texttt{window.navigator} JavaScript object which is available to any website that a user visits with JavaScript enabled.
This information is explained in Table~\ref{tab:navigator-props}.

\begin{table}[h]
\centering
\begin{tabular}{| l | r |}
    \hline
    \textbf{Navigator Property} & \textbf{Description} \\ \hline
    \texttt{appVersion} & {Returns the version of the browser} \\ \hline
    \texttt{cookieEnabled} & {Returns whether or not cookies are enabled} \\ \hline
    \texttt{javaEnabled()} & {Returns whether or not Java is enabled} \\ \hline
    \texttt{language} & {Returns the language of the browser} \\ \hline
    \texttt{platform} & {Returns the browser platform / operating system} \\ \hline
    \texttt{product} & {Returns the product name of the browser engine} \\ \hline
    \texttt{userAgent} & {Returns the User Agent of the browser} \\
    \hline
\end{tabular}
\caption{Table explaining what the relevant \texttt{window.navigator} properties reveal}
\label{tab:navigator-props}
\end{table}

Whereas these characteristics are convenient to access with JavaScript and readily supplied by the browser, there's also a huge amount of additional information which can be retrieved.
The first of which is detecting what fonts the user has installed.

Browsers and JavaScript don't supply any functions for retrieving the full list of fonts that the user has installed.
There is however a method that allows the visited website to detect what fonts are installed by querying a large number of fonts one-by-one.
The method works by creating a HTML \texttt{<span>} element and set the font to a font known to be installed.
After this is done, JavaScript is used to change the CSS of the text so that the font family is different.
The size of the \texttt{<span>} element will be different before and after the change if the probed font is installed, illustrated in Figure~\ref{fig:font-change}.
A huge list of fonts can be tested for using this method, bypassing the inherent security feature of browsers not allowing full font enumeration.

\begin{figure}[h]
\caption{Showing the difference in \texttt{<span>} width when the font is changed}
\includegraphics[scale=0.8]{font-change}
\centering
\label{fig:font-change}
\end{figure}


- Other JavaScript FP methods incl.\ canvas, audionode, math.tan.

- Flash methods incl.\ platform (http://stackoverflow.com/questions/3894488/is-there-anyway-to-detect-os-language-using-javascript), fonts

- Java shizzle

- Effectiveness
