\chapter{Results and Insights}

There were are some promising insights gained over the course of the project.
Table~\ref{tab:results} indicates the difference in how common certain browser characteristics are when Flash and Canvas Fingerprinting are available compared to when they are disabled.
A larger percentage indicates a more common value for a particular characteristic.
A Sony Vaio laptop with the OpenSuse Linux distribution was used for testing, running Firefox as the browser of choice.
The figures are calculated using \url{browserprint.info}, which provides how common each fingerprint characteristic is relative to all other users that have retrieved their fingerprint using the service.
The other techniques that were used were not immediately successful and either will not help to mitigate fingerprinting or require further research and development to improve.
This chapter will discuss the various parts of the browser that expose information, the methods used to try to restrict them and how successful they were.

\begin{table}[h!]
\centering
\begin{tabular}{| p{6cm} | p{4cm} | p{4cm} |}
    \hline
    \textbf{Browser Characteristic} & \textbf{Frequency of value without add-on and settings} & \textbf{Frequency of value with add-on and settings} \\ \hline
    \texttt{Platform (Flash)} & {0.015\%} & {38.46\%} \\ \hline
    \texttt{Screen Size (Flash)} & {10.5\%} & {38.46\%} \\ \hline
    \texttt{Language (Flash)} & {25.38\%} & {38.46\%} \\ \hline
    \texttt{System Fonts (Flash)} & {0.022\%} & {38.46\%} \\ \hline
    \texttt{Canvas Fingerprint} & {0.068\%} & {2.729\%} \\
    \hline
\end{tabular}
\caption{Frequency of browser characteristics without protection vs.\ with protection}
\label{tab:results}
\end{table}

\subsubsection{HTTP Request Headers}

Although in an early stage of development, a few of the HTTP request headers were edited, after further research it was clear that it was not effective for more complex fingerprinters.
The spoofing of the User Agent does not work because fingerprinters are able to query various JavaScript functions and objects available across different browsers and through process of elimination deduce what browser, browser version and platform are being used.
An example of this is querying for the \texttt{ActiveXObject} in the \texttt{window} object, which is available only on the Internet Explorer web browser \citep{activeX}.
The only defence against this is to disable JavaScript altogether.
Unfortunately, this has a big impact on usability due to approximately 94.5\% of all websites use JavaScript \citep{w3-javascript}.
The other headers that could be altered are the HTTP\_ACCEPT headers, particularly the language header, and the date header.
Once again changing these have consequences for the usability of many websites.

\subsubsection{Navigator}



\subsubsection{Plugins}

Text

\subsubsection{Font Fingerprinting}

Text

\subsubsection{Flash}

Text

\subsubsection{Is there a reasonable way to fight it as a user?}

Text

