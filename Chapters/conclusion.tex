\chapter{Conclusion}

As outlined in the previous chapter, in terms of fully defending against browser fingerprinting, the project can not be called a success.
The final product mitigates some fingerprinting techniques, but according to the various fingerprint testing websites used, the browser configuration on the test computer is still a unique one.
However, the final product does somewhat mitigate fingerprinting.
Perhaps with a much larger sample size of fingerprints, the developed product could make a difference and make the browser non-unique.
Along with the positive outcome of mitigating some of the fingerprinting methods, the project also functioned as a great learning experience and gave an enormous insight into how effective browser fingerprinting is for identifying users and just how difficult a job it is to fight it as a regular browser user.

- what could have been done better
  - more in-depth research to begin with, specifically how advanced fingerprinters are
  - canvas fingerprinting not breaking canvases altogether
  - preventing audio fingerpringing

- what needs to be done to prevent fingerprinting
  - browsers more involved in protecting their users
    - fonts
    - canvas
    - audio
    - built more for protecting the user instead of providing easy information to websites
  - homogenous JavaScript APIs across different browsers (probably impossible)

future work

- prevention of audio fp in add-on

- observe fingerprinting `from the other side' so that more insight can be gained

